%%%%%%%%%%%%%%%%%%%%%%%%%%%%%%%%%%%%%%%%%%%%%%%%%%%%%%%%%%%%%%%%%%%%%%%%%%%%%%%
%                          LaTeX report template                              %
%              © 2010 Valentin Roussellet (louen_AD_palouf.org)               %
%    Distributed under the terms of WTFPL v2, see http://sam.zoy.org/wtfpl    %
%                 Version 2 for modern LaTeX, 13 of May 2023                  %
%                       Compile with XeLaTeX or LuaLaTeX                      %
%%%%%%%%%%%%%%%%%%%%%%%%%%%%%%%%%%%%%%%%%%%%%%%%%%%%%%%%%%%%%%%%%%%%%%%%%%%%%%%
\documentclass[a4paper, 10pt]{article}

% ============================== PREAMBLE =====================================
% Language---------------------------------------------------------------------
\usepackage[english]{babel}

% Margins----------------------------------------------------------------------
\usepackage{geometry}
\geometry{hmargin=4.0cm, vmargin=2.0cm }

% Useful packages--------------------------------------------------------------
\usepackage{graphicx}                           % including  images
\usepackage{xcolor}                             % color texts
\usepackage{subcaption}                         % caption subfigures
\usepackage[tight]{shorttoc}                    % for a short summary
\usepackage[colorlinks=true,
            linkcolor=blue,
            citecolor=blue,
            urlcolor=blue,
            bookmarksopen=false]{hyperref}      % pdf links in the document
%\usepackage{listings}                          % for source code inclusion

%\usepackage{fontspec}

% Mathematical symbols---------------------------------------------------------
\usepackage{amsmath}
\usepackage{amsfonts}
\usepackage{amssymb}
\usepackage{amsfonts}               % math fonts (mathbb, mathcal, etc.)
\usepackage{mathrsfs}               % support for both mathsrc and mathcal
\usepackage{amsthm}                 % theorem environment
\usepackage{mathtools}              % shortintertext

\usepackage{unicode-math} % Must be loaded after ams packages

\usepackage[backend=biber,      % use biber as our bib tool
            style=alphabetic,   % similar to alpha with short keys [XYX+12]
            maxcitenames=2,     % Cite at most two authors before "et al."
            maxbibnames=99,     % Full names in bibliography
            refsection=section, % only show the bibliography of the section
            doi=false, isbn=false, % do not show DOI or ISBN
            natbib=true]{biblatex} % Let us have citet and citep from natbib
\renewcommand\mkbibnamefamily[1]{\textsc{#1}} % Use small caps for names
\AtBeginBibliography{\small} % Smaller text for bibliography

\addbibresource{divisors.bib}

\usepackage{lscape}
\usepackage{csquotes}
\usepackage{cleveref}
\usepackage{textcomp}
\usepackage{gensymb}



% Typesetting ----------

\allowdisplaybreaks                 % allow equations page break
\linespread{1.0}                    % Line spacing

% Main font is Tex Gyre Pagella / Fira Sans / Fira Code
\defaultfontfeatures{Scale=MatchLowercase}
\setmainfont[Ligatures=TeX]{Tex Gyre Pagella}
\setsansfont[Ligatures = TeX,
             FontFace = {sb}{n}{Fira Sans SemiBold},
             FontFace = {sb}{i}{Fira Sans SemiBold Italic}
]{Fira Sans}
\setmonofont[FakeStretch=0.9]{Fira Code}

% Select math fonts to match
\setmathfont[Ligatures=TeX]{TeX Gyre Pagella Math}

% Enable semi-bold
\DeclareOldFontCommand{\sbseries}{\fontseries{sb}\selectfont}{\mathbf}
\DeclareTextFontCommand{\textsb}{\sbseries}

% Use XITS maths for different mathcal and mathscr symbols
\setmathfont[range={\mathcal,\mathbfcal},StylisticSet=1]{XITS Math}
\setmathfont[range=\mathscr]{XITS Math}



\numberwithin{equation}{section} % number equations per section

% Math and other comnmands
% Mathematical Notations and useful command in latex

% Geometry and algebra

% Point: bold lower case letter
\newcommand{\pt}[1]{\boldsymbol{\mathbf{#1}}}

% Common shorthand for vertices
\newcommand{\vi}{\pt{v}_i}
\newcommand{\vj}{\pt{v}_j}

% Vector : bold lower case letter with arrow
\renewcommand{\vec}[1]{\protect\overrightarrow{\boldsymbol{\mathbf{#1}}}}

% Norm of a vector using single vertical bars
\newcommand{\norm}[1]{\left\lVert #1 \right\rVert}

% Matrix : bold capital letter
\newcommand{\mat}[1]{\boldsymbol{\mathbf{#1}}}

% Inverse, transpose and inverse transpose
\newcommand{\invmat}[1]{\boldsymbol{\mathbf{#1}}^{-1}}
\newcommand{\tmat}[1]{\boldsymbol{\mathbf{#1}}^{\mathrm{T}}}
\newcommand{\tinvmat}[1]{\boldsymbol{\mathbf{#1}}^{\mathrm{-T}}}

% Operators 
\DeclareMathOperator{\supp}{supp} % Support of a function
\DeclareMathOperator{\diag}{diag} % Diagonal matrix
\DeclareMathOperator{\proj}{proj} % Projection


% Optimization
\DeclareMathOperator*{\argmin}{arg\,min}
\DeclareMathOperator*{\argmax}{arg\,max}
\newcommand{\subjectto}{\mathrm{s.t.\;}}

\DeclareMathOperator*{\relu}{ReLU}

% Differential operator : straight d (useful for integrals)
\newcommand{\dd}{\mathrm{d}}

% Jacobian (matrix J with index)
\newcommand{\Jac}[1]{\mat{J}_{#1}}

% Probablilty theory

% Expectation
\newcommand{\E}{\mathbb{E}}

% Variance
\newcommand{\V}{\mathbb{V}}

% Kulback-Leiter distance
\newcommand{\KLD}[2]{\mathrm{KL}\left[ #1 \Big|\Big| #2 \right]}

% Other latex shortcuts

% Possessive citation (e.g. "In Einstein's [Ein05] work")
\newcommand{\citepos}[1]{\citeauthor{#1}'s \cite{#1}}

% Subfigure reference
\newcommand{\subfigref}[1]{(Figure~(\subref{#1}))}



\setlength\parindent{0pt} % Removes all indentation from paragraphs - comment this line for an assignment with lots of text

% ============================== FRONT MATTER =================================

\title{Dice and divisors}
\author{Valentin \textsc{Roussellet}}
\date{\today}

\begin{document}
\maketitle

\theoremstyle{definition}
\newtheorem*{problem}{Problem}
\begin{problem}
Rolling two 100-sided dice, what are the odds of getting numbers that share exactly three divisors?
\end{problem}

Let $N \geq 1$ be the dice number and $k > 0$ the number of shared factors.
How many pairs $P_k(N)$ of integers $ 1 \leq n \leq N$ share exactly $k$ factors?

\paragraph{Preliminaries}
In the following discussion, a number's divisors include 1 and itself.
For example, the factors of 6 are 1,2,3, and 6.
For this reason, any two numbers will always have at least 1 as a common divisor.

For any number $N$ we denote by $s_k(N)$ how many integers $1 \leq n \leq N$
share k factors with $N$. For example for $N = 10$ and $k=2$: 10 shares factors (1,2) wih
four integers (2,4,6, and 8) and shares factors (1,5) with 5.
Any other odd integer will not share any factor other than 1 with 10,
and 10 itself shares four with itself (1,2,5, and 10). Thus $\sigma(10,2) = 5$

When studying common factors in pairs of integers, a useful
number is their their greatest common divisor (GCD) and its divisors.
\newtheorem{lemma}{Lemma}
\begin{lemma}
    Let $x$ and $y \in \mathbb{N}$.
    $x$ and $y$ share exactly $k$ factors if and only if their GCD
    has exactly $k$ factors.
    \label{thm:div}
\end{lemma}
\begin{proof}
    This comes directly from the fact that any common divisor of $x$ and $y$
    must be a divisor of $\gcd(x, y)$.
\end{proof}

In the following, we will prove formulas for $k=1,2,3$, by induction on $N$.
Making use of the idea that when going from $N-1$ to $N$, we must add to the previous count $P(N-1,k)$
the pairs of the form $(x,N)$ and $(N,y)$ for any $x,y \leq N$, of which there are $s(N,k)$
However, if $(N,N)$ is a valid pair, then we must substract 1.
We thus derive the following recurrence relation:
\begin{lemma}
    Let $N > 1$ and $ 1 \leq k < N$.
    \[ P_k(N) = P_k(N-1) + 2\, s_k(N) -
    \begin{dcases*}
        1 & if $(N,N)$ shares $k$ factors \\
        0 & otherwise
    \end{dcases*}
\]
    \label{thm:rec}
\end{lemma}


\section*{Case $k = 1$}
In this case, $x$ and $y$ share exactly one factor: 1.
By using Lemma \ref{thm:div}, this is equivalent to $\gcd(x,y)=1$:~in other words $x$ and $y$ share one factor
if and only if they are coprime.

We introduce $\phi$, Euler's totient function, where $\phi(n)$ counts coprimes less than or equal to $n$.
yielding immediately:
\[
    s_1(N) = \phi(N)
\]

We can then state the result for $k=1$:
\newtheorem{thm}{Theorem}
\begin{thm}
\[
    P_1(N) = 2 \left( \sum_{n=1}^N \phi(n)  \right)- 1
\]
\end{thm}
\begin{proof}
    We prove this theorem by induction.

    Firstly, we check that $P_1(1) = 1$, that is the formula is true for $N = 1$,
    as there is exactly one pair of coprime numbers: $(1,1)$ and
    $\phi(1) = 1$

    Next, assume the formula holds for $N -1, \,N > 1$.
    Using lemma \ref{thm:rec}, to count the additional pairs, we have
     \[ P_1(N) = P_1(N-1) + 2 \phi(N) \]
    Note that the pair $(N, N)$ is never counted twice, since $N>1$ cannot be coprime with itself.
    Therefore, substituting $P_1(N-1)$ from our inductive hypothesis
    \[
        P_1(N) = 2 \left( \sum_{n=1}^{N} \phi(n)  \right) - 1
    \]

\end{proof}

\section*{Case $k=2$}

\begin{lemma}
Let $x$ and $y$ be two positive integers. $x$ and $y$ share exactly two divisors
if and only if their GCD $p$ is prime and $\gcd(\frac{x}{p}, \frac{y}{p}) = 1$
\label{thm:prime}
\end{lemma}
\begin{proof}
Suppose $x$ and $y$ share exactly two divisors: 1 and $p$.
By lemma \ref{thm:div}, this means that $\gcd(x,y) = p$
has exactly two divisors, so $p$ is a prime number.

If we denote $a = \frac{x}{p}$ and $b= \frac{y}{p}$, we show that $a$ and $b$ must be coprime.
Since $\gcd(a,b)  \mid  a$ and $a \mid x$ then $\gcd(a,b)  \mid  x$; similarly $\gcd(a,b)  \mid  y$.
Therefore $\gcd(a,b)$ is a common factor of $x$ and $y$, and thus is either 1 or $p$.
However it cannot be $p$; otherwise, $p^2$ would divide $x$ and $y$.

Conversely, if $p$ is a prime and $a$ and $b$ are two coprime positive integers,
$pa$ and $pb$ share exactly two factors, 1 and $p$.
Suppose there is a third common factor $q \neq 1$. Since $p$ is prime then $q  \mid  a$ and $q  \mid b$,
so $q  \mid  \gcd(a,b)$, which is contradicts $a$ and $b$ being coprime.
\end{proof}

Denoting $\pi(n)$ as the function counting the number of primes less than $n$,
we can write the formula for $P_2$
\begin{thm}
\[
    P_2(N) = 2\left(\sum_{n=1}^N \sum_{p \mid n}\phi(\frac{n}{p})\right) - \pi(N)
\]
where $\sum_{p \mid n}$ means summing over all primes that divide $n$
\end{thm}

\begin{proof}
Similarly to the previous section, we prove this formula by induction.
For $N=1$ there are no pairs that share two factors since $(1,1)$ has only one factor.
Thus $P_2(1) = 0$.

Next, we assume the formula holds for $N-1$ with $N > 1$.
To use the recurrence of lemma \ref{thm:rec} we must express $s_2(N)$

By lemma \ref{thm:prime} $(x,N)$ share two factors if their GCD is a prime number
$p$, thus $p$ must be a prime factor of $N$.

The numbers $x$ sharing factors $(1,p)$ with $N$ are numbers of the form $p \times a$ where
$a$ must be coprime with $\frac{N}{p}$.
For a given prime divisor $p$ of $N$ there are exactly $\phi(\frac{N}{p})$ numbers coprime
with the quotient.
\[ s_2(N) = \sum_{p  \mid  N} \phi(\frac{N}{p})  \]

To apply the recurrence relation we must check the pair $(N,N)$.
It shares two factors if and only if $\gcd(N,N) = N$ is a prime
number. It follows that we doubled counted $(N,N)$ if and only if $N$ is prime.

We distinguish the two cases:
\begin{itemize}
    \item if $N$ is not prime, then $(N,N)$ shares more than two factors,
    so this pair will not be double-counted; and we have
    \[
        P_2(N) = P_2(N-1) +  2 \sum_{p  \mid  N} \phi(\frac{N}{p})
    \]
    Since in this case, $\pi(N) = \pi(N-1)$ the formula holds.

    \item if $N$ is prime, then it has only one prime factor, (which is $N$ itself).
    \[ \sum_{p  \mid  N} \phi(\frac{N}{p}) = \phi(1) = 1 \]
    in this case, $\pi(N) = \pi(N-1) + 1$ so we can write
    \begin{align*}
        P_2(N)  &= P_2(N-1) + 1\\
                &= P_2(N-1,2) +  2 \sum_{p  \mid  N} \phi(\frac{N}{p}) - 1 \\
                &=  2\left(\sum_{n=1}^N \sum_{p \mid n}\phi(\frac{n}{p})\right) -\pi(N-1) - 1\\
                &=  2\left(\sum_{n=1}^N \sum_{p \mid n}\phi(\frac{n}{p})\right) -\pi(N)\\
    \end{align*}
\end{itemize}

\end{proof}

\section*{Case $k=3$}

\begin{lemma}
    Let $(x,y)$ be two positive integers. They share exactly three divisors if and only if
    their GCD  is the square of a prime $p$ and if $\frac{x}{p^2}$ and $\frac{y}{p^2}$
    are coprime.
    \label{thm:primesq}
\end{lemma}
\begin{proof}
Suppose $x$ and $y$ share exactly three divisors. By lemma \ref{thm:div} their $gcd$
has exactly three divisors.
This implies that $g = \gcd(x,y) >1$, and therefore $x$ and $y$ are not coprime.
Since $g > 1$ we already know two distinct divisors of $g$ : 1 and $g$ itself.
This means there is a third divisor $p$ such that $1 < p < g$.

We then show that $p$ is prime. If $p$ was composite, then any non trivial divisor of
$p$ would also divide $g$, which contradicts $g$ having only $\{1,p,g\}$ as divisors.

Since $p  \mid  g$, so does $\frac{g}{p}$; it must be one of the three divisors of $g$.
It can neither be $1$ or $g$, since $p \neq 1$ and $p \neq g$. Thus $\frac{g}{p} = p$
and $g = p^2$.

Similarly to the $k=2$ case, we define the quotients $a =\frac{x}{p^2}$ and $b = \frac{y}{p^2}$.
$x$ and $y$ are both divisible by $\gcd(a,b)$, therefore it also must be in $\{1,p,p^2\}$.
But $p \nmid  \gcd(a,b)$. If that was the case then $p^3$ would be a divisor of $x$ and $y$.
So $\gcd(a,b)$ can neither be $p$ nor $p^2$, thus it must be 1, and $a$ and $b$ are coprime.


Conversely, if $p$ is a prime, and $a,b$ two coprime integers, $x = ap^2$ and $y = ap^2$
share three factors : 1 $p$, and $p^2$.
\end{proof}
\newtheorem{cor}{Corollary}
\begin{cor}
    If an integer $x$ is squarefree (i.e. not divisible by any square of a prime), then there is no integer $y$ that
    shares exactly three divisors with $x$.
    \label{thm:sqf}
\end{cor}


Denoting $Q(n)$ the number of squarefree integers between 1 and $n$, we get
an expression for $P_3$.

\begin{thm}
\[
    P_3(N) = 2\left(\sum_{n=1}^N \sum_{p^2 \mid n}\phi(\frac{n}{p^2})\right) - \pi(\sqrt{N})
\]
where $\sum_{p^2 \mid n}$ means summing over all primes whose square divide $n$.
If $n$ is squarefree, this sum is 0.
\end{thm}
\begin{proof}
 We first check the formula is valid for $N=1$, as $P_3(1) = 0$; since no square of prime divides 1.

Next, we again assume the formula for $P_3$ is valid for $N-1$ with $N>1$.
To apply lemma \ref{thm:rec} we must compute $s_3(N)$.
By lemma \ref{thm:primesq}, $(x,N)$ share three factors if and only if their GCD is
a square of prime.
For each prime $p$ such as $p^2 \mid N$, the numbers $x$ sharing factors $(1,p,p^2)$ with n
are of the form $p^ \times a$ where $a$ must be coprime with $\frac{N}{p^2}$.
For a given square of prime $p^2$ that divides $N$, there are exactly
$\phi(\frac{N}{p^2})$ numbers coprime with the quotient.
Thus
\[
    s_3(N) = \sum_{p^2 \mid n}\phi(\frac{n}{p^2})
\]
Now,  we double-count the pair $(N,N)$ when $N$ has exactly three factors.
In this case $N = p^2$ is the square of a prime.
Since $\pi(\sqrt{N})$ increments only when $N$ is a square of a prime,
this gives us the formula.
\end{proof}

\section*{Probability and asymptotical value}

To answer the original question, we must give a value for
the probability of a pair sharing $k$ divisors drawn from two
independent uniform distriutions: $\frac{P_k(N)}{N^2}$

The totient summatory function $\Phi(N) = \sum_{n=1}^N \phi(n)$
has an asymptotic equivalent \cite{totsum}
\[\Phi(N) ~ \frac{1}{2} \frac{1}{\zeta(2)} N^2  + O(N \log N )\]
where $\zeta$ is the Riemann zeta function.
This immediately yields
\begin{align*}
    \lim_{N \rightarrow \infty} \frac{P_1(N)}{N^2}
    &=  \frac{1}{\zeta(2)} \\
    &=  \frac{6}{\pi^2} \\
    &\approx 0.6079
\end{align*}

\printbibliography

\end{document}


