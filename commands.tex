% Mathematical Notations and useful command in latex

% Geometry and algebra

% Point: bold lower case letter
\newcommand{\pt}[1]{\boldsymbol{\mathbf{#1}}}

% Common shorthand for vertices
\newcommand{\vi}{\pt{v}_i}
\newcommand{\vj}{\pt{v}_j}

% Vector : bold lower case letter with arrow
\renewcommand{\vec}[1]{\protect\overrightarrow{\boldsymbol{\mathbf{#1}}}}

% Norm of a vector using single vertical bars
\newcommand{\norm}[1]{\left\lVert #1 \right\rVert}

% Matrix : bold capital letter
\newcommand{\mat}[1]{\boldsymbol{\mathbf{#1}}}

% Inverse, transpose and inverse transpose
\newcommand{\invmat}[1]{\boldsymbol{\mathbf{#1}}^{-1}}
\newcommand{\tmat}[1]{\boldsymbol{\mathbf{#1}}^{\mathrm{T}}}
\newcommand{\tinvmat}[1]{\boldsymbol{\mathbf{#1}}^{\mathrm{-T}}}

% Operators 
\DeclareMathOperator{\supp}{supp} % Support of a function
\DeclareMathOperator{\diag}{diag} % Diagonal matrix
\DeclareMathOperator{\proj}{proj} % Projection


% Optimization
\DeclareMathOperator*{\argmin}{arg\,min}
\DeclareMathOperator*{\argmax}{arg\,max}
\newcommand{\subjectto}{\mathrm{s.t.\;}}

\DeclareMathOperator*{\relu}{ReLU}

% Differential operator : straight d (useful for integrals)
\newcommand{\dd}{\mathrm{d}}

% Jacobian (matrix J with index)
\newcommand{\Jac}[1]{\mat{J}_{#1}}

% Probablilty theory

% Expectation
\newcommand{\E}{\mathbb{E}}

% Variance
\newcommand{\V}{\mathbb{V}}

% Kulback-Leiter distance
\newcommand{\KLD}[2]{\mathrm{KL}\left[ #1 \Big|\Big| #2 \right]}

% Other latex shortcuts

% Possessive citation (e.g. "In Einstein's [Ein05] work")
\newcommand{\citepos}[1]{\citeauthor{#1}'s \cite{#1}}

% Subfigure reference
\newcommand{\subfigref}[1]{(Figure~(\subref{#1}))}

